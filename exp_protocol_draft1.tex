\documentclass[12pt]{article}
\usepackage[a4paper]{geometry}
\usepackage{microtype}
\usepackage{titling}
\usepackage{helvet}
\setlength{\droptitle}{-4cm}

\title{Proposed Experimental Protocol}
\author{Dermott McMorrough}
\date{\today}


\begin{document} 
\maketitle

From June 24$^{th}$ to September 18$^{th}$ 2016, the Science Gallery Dublin will host an exhibition entitled ``\textsc{Seeing}''. During this time, I hope to take part in a research residency, where I set up and supervise a series of experiments in the gallery for a number of weeks. The overall aim is to gain a better understanding of the links between $\alpha$-wave frequency, critical flicker fusion (CFF), and temporal acuity. 
The experiment should last less than five minutes, as not to keep subjects too long, and to maximise the number of subjects examined. At the start of the experiment, a measure of $\alpha$-wave frequency will be taken before moving onto the tests proposed here.

\section{Flicker Fusion Threshold}
To establish a subject's flicker fusion threshold, they will be tested on their ability to differentiate between flickering and steady stimuli. They will be shown a pair of stimuli on a monitor, and asked to identify which of the two stimuli (if any) is flickering. The frequency of the flicker will be adjusted with each reptition, with the goal being to identify the freqency at which the flickering image appears steady to the observer. This can either be done by randomly adjusting the flicker rate or by progressively adjusting the rate in a step-wise manner as to narrow-in on the threshold rate. \textit{The exact method of identifying the threshold needs to tested in a prototype set up.}
As CFF is a statistical, rather than absolute value, there is a range of frequencies within which flicker sometimes will be seen and sometimes will not be seen, the threshold is the frequency at which flicker is detected on 50\% of trials. It is ncessary to take a number of measurements of a person's flicker threshold in order to apporximate a realistic value.  

\section{Possible Tests}
Once we have a measurement of CFF for an individual, we can use the remaining time to conduct a few short psychometric tests. This is really where I am looking for input.  

So far discussions have centred around experiments including:

\begin{itemize}
\item ``Join the Dots'' - need more information on this. 
\item The ``Wagon Wheel Effect'', (alternatively, stagecoach-wheel effect, stroboscopic effect) is an optical illusion in which a spoked wheel appears to rotate differently from its true rotation. 
\item The ``Michotte Effect'',  described as one dot moves across the screen and another dot moves away, which is designed to test our ability to discern cause and effect.  
\end{itemize}
In a short time frame of five minutes, I am looking to figure out what are the most suitable tests to deploy to answer our questions.



%I have not done this particular experiment myself, but what I have done in relation to time is:
%Time discrimination  - i.e. Two time durations (1000 and 1400msecs or 50 and 75msecs), just noticeable difference calculated
%Time estimation  and time production (self explanatory)
%These time tasks provide info about the basic timing processes.

\section{Other considerations}
As with all exhibitions in the Science Gallery, I have been asked to consider how this experimental set-up will be experienced by the partons of the gallery. The narrative of CFF affecting your ability to play baseball went down well when I met with exhinition staff, and so I discussed the idea of recording baseball piches at different speeds and having a video representing the CFF phenomenon. 

In practical terms, I need to apply for level 2 ethical clearance to conduct this experiment on anyone under 18, which will either go through the science gallery directly or through a school-level ethics board. 
The personal data of any subject collected will be subject to data protection laws, and so some considerration will go into what questions we want to answer from the CFF measurements (such as age, gender etc.).
I have also read into the risk rergarding Epilepsy, and am happy that it will not be too much of an issue. This, of course, will be the decision of an ethics board.

\end{document}